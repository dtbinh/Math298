\documentclass[mathserif,notes]{beamer}

\usepackage{pdfsync} % works only on TeXShop, comment otherwise
\usepackage{pgf}
\usepackage{multimedia} % to embed movies in the PDF file
\usepackage{xcolor}
\usepackage{graphicx}
\usepackage{comment}
\usepackage{pgfpages}
\usepackage{topcapt}
\usepackage{booktabs}

% for the tikz pictures
\usepackage{geometry}
%\geometry{hmargin=1cm,vmargin=1cm}
\usepackage{tikz}
\def\width{5}
\def\hauteur{5}
\mode<presentation>
{
  \usetheme{Madrid}   %  \usetheme{Warsaw}
  % Note: this next command makes nice slides, but Adobe doesn't like it (you get blank pages)
  \setbeamertemplate{background canvas}[vertical shading][bottom=white,top=structure.fg!15]
  %\setbeamertemplate{footline}{\pgfuseimage{UCM-logo}}

  %\usecolortheme{seahorse}
  \usecolortheme{beaver}
 % \setbeamercovered{transparent}
}
%\usepackage[T1]{fontenc}
% For printing purposes only
%\pgfpagesuselayout{2 on 1}[letterpaper,border shrink=5mm,portrait]
\pgfpagesuselayout{resize to}[letterpaper,border shrink=5mm,landscape]  % print out full size on paper

\usepackage{amssymb,amsmath}
\hypersetup{%
 pdftitle={Math298 Lecture 8},
 pdfauthor={Juan Meza(UC Merced)},
 pdfsubject={Fundamental Concepts in Computational and Applied Mathematics},
 pdfkeywords={Optimization and Nonlinear Equations}
 }
 
\title[Math 298 Lecture 8: Optimization]{Fundamental Concepts in \\ Computational and Applied Mathematics}  \subtitle{}
\author[Juan Meza]{Juan Meza \\ School of Natural Sciences \\ University of California, Merced}
\date[Fall 2014]{Fall 2014}
\institute[UC Merced]

\pgfdeclareimage[height=0.35cm]{UCM-logo}{UCM-logo}
% \logo{\pgfuseimage{UCM-logo}}
\pgfdeclareimage[width=0.7\textwidth]{UCM-topo}{UCM-topo}
%% figures

% My LaTeX definitions with some common abbreviations
%




% Common abbreviations
% (Remember to put '\ ' after if an interword space is
%  desired rather than end-of-sentence space. Same for '.etc)' ).
\newcommand{\eg}{{\em e.g.}}		% e.g.
\newcommand{\ie}{{\em i.e.}}		% i.e.
\newcommand{\etc}{{\em etc.}}		% etc.
\newcommand{\vs}{{\em vs.}}		% vs.



\newcommand {\DS} {\displaystyle}
\newcommand{\eref}[1]{{\rm{(\ref{#1})}}}
\newcommand{\tref}[1]{{\rm{\ref{#1}}}}


\def\frechet{Fr\'echet\ }
\def\more{Mor\'e\ }
\def\fl{{\sf fl}}
\def\flop{{\sf flop} }
\def\flops{{\sf flops} }
\def\matlab{{\sc Matlab} }
\def\Matlab{\matlab}



% Mathematical Symbols

\newcommand{\deq}{\raisebox{0pt}[1ex][0pt]{$\stackrel{\scriptscriptstyle{\rm def}}{{}={}}$}}


\newcommand {\half} {\mbox{$\frac{1}{2}$}}  %machine epsilon
\newcommand{\set}[2]{\left\{ #1 \;:\; #2 \right\}}


\newcommand {\macheps} {\mathbf{\epsilon}}  %machine epsilon
\newcommand {\real} {\mathbb{R}}
\newcommand {\nat} {\mathbb{N}}
\newcommand {\compl} {\mathbb{C}}
\newcommand {\diag} {{\mbox{diag}}}
\newcommand {\meas} {{\mbox{meas}}}
\newcommand {\mspan} {{\mbox{span}}}

\newcommand {\pdx} {\frac{\partial}{\partial x}}
\newcommand {\pdt} {\frac{\partial}{\partial t}}
\newcommand {\pdxx} {\frac{\partial^2}{\partial x^2}}
\newcommand {\dx} {\frac{d}{d x}}
\newcommand {\dt} {\frac{d}{d t}}
\newcommand {\dxx} {\frac{d^2}{d x^2}}

\newcommand {\bA} {\mbox{\boldmath $A$}}
\newcommand {\bB} {\mbox{\boldmath $B$}}
\newcommand {\bF} {\mbox{\boldmath $F$}}
\newcommand {\bI} {\mbox{\boldmath $I$}}
\newcommand {\bP} {\mbox{\boldmath $P$}}
\newcommand {\bY} {\mbox{\boldmath $Y$}}
\newcommand {\bZ} {\mbox{\boldmath $Z$}}
\newcommand {\ba} {\mbox{\boldmath $a$}}
\newcommand {\bb} {\mbox{\boldmath $b$}}
\newcommand {\bd} {\mbox{\boldmath $d$}}
\newcommand {\bff}{\mbox{\boldmath $f$}}
\newcommand {\bk} {\mbox{\boldmath $k$}}
\newcommand {\bp} {\mbox{\boldmath $p$}}
\newcommand {\bs} {\mbox{\boldmath $s$}}
\newcommand {\bu} {\mbox{\boldmath $u$}}
\newcommand {\bv} {\mbox{\boldmath $v$}}
\newcommand {\bw} {\mbox{\boldmath $w$}}
\newcommand {\bx} {\mbox{\boldmath $x$}}
\newcommand {\by} {\mbox{\boldmath $y$}}
\newcommand {\bz} {\mbox{\boldmath $z$}}
\newcommand {\blambda} {\mbox{\boldmath $\lambda$}}
\newcommand {\btau} {\mbox{\boldmath $\tau$}}

%
% JCM Quantum Chemistry commands
%
\newcommand{\bra}[1]{\langle #1|}
\newcommand{\ket}[1]{|#1\rangle}
\newcommand{\braket}[2]{\langle #1|#2\rangle}
% Example of usage
%\[
%\ket{\Psi}=\sum_{i}\ket{\phi_i}\braket{\phi_i}{\Psi}
%\]

%
% Other symbols
%
\newcommand {\grad}{\nabla}
\newcommand {\hess}{\nabla^2}
\newcommand {\condA}{\kappa(A)}


\newtheorem{algorithm}{Algorithm}

%

% \boxfig{pos}{wid}{text}:  A figure with a box around it
%
% pos	the usual figure placement arg: eg. htbp
% wid	the width of the figure, in some units: eg. 5in
% text	the contents of the figure, including picture/caption/label/etc
%
\newlength{\boxwidth}
\newcommand{\boxfigure}[3]{
	\begin{figure}[#1]
		\setlength{\boxwidth}{#2}
		\addtolength{\boxwidth}{.1in}

		\centering
		\framebox[\boxwidth]{
			\begin{minipage}{#2}
			#3
			\end{minipage}
		}
	\end{figure}  
}

% use \fullboxwidth for arg 2 of boxfigure to get box of size \textwidth


% \boxalg{pos}{text}:  An algorithm with a box around it
%
% title  the name of the algorithm
% text   the contents of the figure, including picture/caption/label/etc
%
\newcommand{\boxalg}[2]{
            \setlength{\boxwidth}{\fullboxwidth}
            \addtolength{\boxwidth}{.1in}
            \vspace{2ex}
            \framebox[\boxwidth][#1]{
                    \vspace{1ex}
                    \begin{minipage}{\fullinboxwidth}
                        #2
                    \end{minipage}
                    \vspace{1ex}
            }
            \vspace{2ex}
}






% see above
\newlength{\fullboxwidth}
\setlength{\fullboxwidth}{\textwidth}
\addtolength{\fullboxwidth}{-0.5in}

\newlength{\fullinboxwidth}
\setlength{\fullinboxwidth}{\fullboxwidth}
\addtolength{\fullinboxwidth}{-0.5in}





\def\bigO{\mathcal{O}}
\def\xstar{x_{*}}
\def\Rn{\real^{n}}
\newcommand {\real} {\mathbb{R}}
\newcommand {\grad}{\nabla}
\newcommand {\hess}{\nabla^2}

\definecolor{navy}{RGB}{0,0,128}
\definecolor{forestgreen}{RGB}{34,139,34}
\definecolor{mylightsteelblue}{RGB}{176,196,222}
\definecolor{mylightcyan}{RGB}{180,255,255}
\definecolor{mypaleturquoise}{RGB}{175,238,238}
\definecolor{mylightgoldenrod}{RGB}{250,250,210}
\definecolor{mylightyellow}{RGB}{255,255,224}
\definecolor{mylightsalmon}{RGB}{255,160,122}

\setbeamercolor{blacklightsteelblue}{fg=black,bg=mylightsteelblue}
\setbeamercolor{blackpaleturquoise}{fg=black,bg=mypaleturquoise}
\setbeamercolor{blacklightcyan}{fg=black,bg=mylightcyan}
\setbeamercolor{blacklightgoldenrod}{fg=black,bg=mylightgoldenrod}
\setbeamercolor{blacklightyellow}{fg=black,bg=mylightyellow}
\setbeamercolor{blacklightsalmon}{fg=black,bg=mylightsalmon}
\setbeamercolor{blackcyan}{fg=black,bg=cyan}


\begin{document}

%%%%%%%%%%%%%%%%%%%%%%%%%%%%%%%%%%%%%%%%%%%%%
\frame{

\titlepage

} % end frame

\section{Introduction}
%%%%%%%%%%%%%%%%%%%%%%%%%%%%%%%%%%%%%%%%%%%%%
\frame{\frametitle{Optimization and Nonlinear Equations}

\begin{itemize}
\item Optimization and nonlinear equations at the heart of many science and engineering problems
\item Many of the ideas/methods used in optimization are the same as in nonlinear equations
\item Can divide methods generically into derivative and non-derivative methods
\end{itemize}
For minimization we usually state the problem as: $\min f(x)$ \\
For nonlinear equations we usually state the problem as : Solve $F(x) = 0$
} % end frame


\section{Notation}
%%%%%%%%%%%%%%%%%%%%%%%%%%%%%%%%%%%%%%%%%%%%%
\frame{\frametitle{Nonlinear Equations (NLE)}

Suppose we have:

$$
F: \real^n \rightarrow \Rn
$$
then the problem of solving a set of nonlinear equations is given by:
$$
\mbox{find} \ \xstar \in \Rn  \ \mbox{such that} \ F(\xstar) = 0 \in \Rn
$$
} % end frame

%%%%%%%%%%%%%%%%%%%%%%%%%%%%%%%%%%%%%%%%%%%%%
\frame{\frametitle{Miminization}

Suppose we have:

$$
f: \Rn \rightarrow \real
$$
then the problem of minimization is given by:
$$
\mbox{find} \ \xstar \in \Rn  \ \mbox{such that} \ f(\xstar) \leq f(x), \ \forall x \in \Rn
$$
{\color{red} Remark.} Another variation of this is the {\color{red} constrained version}, i.e.
$$
\min_{x \in \Omega \subset \Rn}  f:  \Rn \rightarrow \real
$$
} % end frame

%%%%%%%%%%%%%%%%%%%%%%%%%%%%%%%%%%%%%%%%%%%%%
\frame{\frametitle{Some important terminology}

\begin{columns}
\column{.5\textwidth}

\begin{itemize}
\item Level sets (curves) are set of points where $f(x) = c$, for some constant; think topographic map 
\item {\color{red} gradient} $g(x) = \nabla f(x) $ is the (column) vector of first derivatives of $f$
\item {\color{red} Jacobian} $J(x) = \nabla F^T (x) $ is the matrix of first derivatives of $F$
\item {\color{red} Hessian} $H(x) = \hess f(x) $ is the matrix of second derivatives of $f$
\end{itemize}
\column{.5\textwidth}
	\pgfputat{\pgfxy(0,-4.0)}{\pgfbox[left,base]{\pgfuseimage{UCM-topo}}}
\end{columns}
} % end frame

%%%%%%%%%%%%%%%%%%%%%%%%%%%%%%%%%%%%%%%%%%%%%
\frame{\frametitle{Motivation}

Main approach to solving nonlinear equations/optimization can be viewed in several ways.
We will consider the {\color{red} model-based} approach

\begin{itemize}
\item Replace your nonlinear problem with a ``model"
\item Solve resulting model system
\item Check for convergence, iterate
\item So for example:
\begin{itemize}
\item Nonlinear equations can be replaced by a linear problem
\item Minimization problem can be replaced by a quadratic model
\end{itemize}
\end{itemize}

} % end frame

\section{Examples}

%%%%%%%%%%%%%%%%%%%%%%%%%%%%%%%%%%%%%%%%%%%%%
\frame{\frametitle{Example 1}

Consider the problem of solving one equation in one unknown, $f(x) = 0$.

First replace $f(x)$ by a linear model $M_c(x)$:
$$
M_c(x) = f(x_c) + f^{\prime}(x_c)(x - x_c).
$$

Solving for $M_c(x) = 0$ we get:
\begin{eqnarray*}
f(x_c) + f^{\prime}(x_c)(x - x_c) &=& 0 \\
x - x_c & = &  - \frac{f(x_c)}{f^{\prime}(x_c)} \\
x &=& x_c  - \frac{f(x_c)}{f^{\prime}(x_c)}
\end{eqnarray*}

\begin{block}{Remark}
You may recognize this as just Newton's method. What does $M_c$ look like in the general $n-$dimensional case?
\end{block}
} % end frame


%%%%%%%%%%%%%%%%%%%%%%%%%%%%%%%%%%%%%%%%%%%%%
\frame{\frametitle{Example 2}

Consider the problem of minimizing a general nonlinear function, $f(x)$.

First replace $f(x)$ by a quadratic model:

$$
m_c(x) = f(x_c) + f^{\prime}(x_c)(x - x_c) + \frac{1}{2} f^{\prime \prime}(x_c)(x - x_c)^2 .
$$

\begin{block}{Remark}
All we have to do now is find the minimum of a quadratic function, i.e. take the derivative of $m_c$ wrt $x$ and set it equal to 0. 
\end{block}
\begin{block}{Remark}
Alternately, we could have just as easily said that we would use our previous approach in Example 1 to solving $f^{\prime}(x) = 0 !$ 
\end{block}
} % end frame

%%%%%%%%%%%%%%%%%%%%%%%%%%%%%%%%%%%%%%%%%%%%%
\frame{\frametitle{Theory}

\begin{itemize}
\item Under some fairly standard assumptions one can show convergence for Newton's method applied to nonlinear equations (NLE)
\begin{itemize}
\item $F(x)$ is continuously differentiable
\item A solution $\xstar$ exists s.t. $F(\xstar) = 0$
\item $J(x)$ is Lipschitz continuous and $J(\xstar)^{-1}$ exists
\end{itemize}

\item For Newton's method one can show {\em quadratic convergence}, i.e.
$$
|| x_{k+1} - \xstar || \leq C || x_{k} - \xstar ||^2 .
$$
\item Variations on Newton's method still have fast convergence, i.e. superlinear convergence.

\end{itemize}

} % end frame


%%%%%%%%%%%%%%%%%%%%%%%%%%%%%%%%%%%%%%%%%%%%%
\frame{\frametitle{Practicalities}

\begin{itemize}
\item Many variations of optimization methods based on problem characteristics
  \begin{itemize}
      \item Convex programming
      \item Derivative-free optimization
      \item Large-scale optimization
      \item Stochastic optimization
      \item $\ldots$
   \end{itemize}
\item Many of the usual theoretical assumptions do not hold for science and engineering problems
   \begin{itemize}
     \item Smoothness/continuity
     \item Availability of derivatives
     \item Infinite precision
   \end{itemize}

\end{itemize}

} % end frame


%%%%%%%%%%%%%%%%%%%%%%%%%%%%%%%%%%%%%%%%%%%%%
\frame{\frametitle{Class Discussion}

Question:  Discuss the advantages and disadvantages to the model-based approach

\vspace{.5in}

Starter questions
\pause
\begin{itemize}[<+->]
\item<2-> What assumptions are we making, relying on?
\item<3-> How do you choose a good model?
\item<4-> What does the model-based approach look like in $\Rn$?
\item<5-> What does convergence mean?
\end{itemize}

} % end frame

%%%%%%%%%%%%%%%%%%%%%%%%%%%%%%%%%%%%%%%%%%%%%
\frame{\frametitle{Advantages/Disadvantages of Newton's Method\footnote{Numerical Methods for Unconstrained Optimization and Nonlinear Equations, J.E. Dennis, Jr. and Robert B. Schnabel, Prentice-Hall}  }
% Requires the booktabs if the memoir class is not being used
\begin{table}[htbp]
   \caption{Newton's method }\centering
   %\topcaption{Table captions are better up top} % requires the topcapt package
   \begin{tabular}{@{} l @{}} % Column formatting, @{} suppresses leading/trailing space
      \toprule
%      \cmidrule(r){1-2} % Partial rule. (r) trims the line a little bit on the right; (l) & (lr) also possible
      {\bf Advantages}\\
      \midrule
	1. $q$-quadratic convergence if $J(\xstar)$ is nonsingular. \\
	2. superlinear convergence for other variations of Newton's method. \\
	3. Exact solution in 1 iteration for affine F. \\
      \midrule
      {\bf Disadvantages} \\
      \midrule
      1. Not globally convergent for all problems. \\
      2. Requires $J(x_k)$ at each iteration. \\
      3. Need solution to system of linear equations at each iteration. \\
      \bottomrule
         \end{tabular}

   \label{tab:NM}
\end{table}

}% end frame

\section{Summary}
%%%%%%%%%%%%%%%%%%%%%%%%%%%%%%%%%%%%%%%%%%%%%
\frame{\frametitle{Summary}
\begin{itemize}
\item Newton-based methods provide some of the most popular and powerful methods for solving nonlinear equations and optimization.
\item Strong theoretical foundation. 
\item Many variations of basic method that take advantage of special structure or problem properties.
\item Most practical science and engineering problems do not lend themselves easily to standard set of assumptions.
\end{itemize}
}% end frame


%%%%%%%%%%%%%%%%%%%%%%%%%%%%%%%%%%%%%%%%%%%%%
\begin{frame}[allowframebreaks]
  \frametitle<presentation>{References}    
  \begin{thebibliography}{10}    

 \beamertemplatebookbibitems
 \bibitem{DeSc}
 J.E. Dennis, Jr. and Robert B. Schnabel
    \newblock {\em Numerical Methods for Unconstrained Optimization and Nonlinear Equations, }
   \newblock SIAM Classics in Applied Mathematics,1987.

 \beamertemplatearticlebibitems
 \bibitem{FlPo}
R. Fletcher and M.J.D. Powell
  \newblock A Rapidly Convergent  Descent Method for Minimization,
 \newblock Comput. J., 6 pp. 163--168, 1963.

 \end{thebibliography}
 \end{frame}

\end{document}
