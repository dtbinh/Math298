




% Common abbreviations
% (Remember to put '\ ' after if an interword space is
%  desired rather than end-of-sentence space. Same for '.etc)' ).
\newcommand{\eg}{{\em e.g.}}		% e.g.
\newcommand{\ie}{{\em i.e.}}		% i.e.
\newcommand{\etc}{{\em etc.}}		% etc.
\newcommand{\vs}{{\em vs.}}		% vs.
\newcommand{\remark}{{\color{red} Remark:} }


\newcommand {\DS} {\displaystyle}
\newcommand{\eref}[1]{{\rm{(\ref{#1})}}}
\newcommand{\tref}[1]{{\rm{\ref{#1}}}}


\def\frechet{Fr\'echet\ }
\def\more{Mor\'e\ }
\def\fl{{\sf fl}}
\def\flop{{\sf flop} }
\def\flops{{\sf flops} }
\def\matlab{{\sc Matlab} }
\def\Matlab{\matlab}

\def\bigO{\mathcal{O}}

% Mathematical Symbols

\newcommand{\deq}{\raisebox{0pt}[1ex][0pt]{$\stackrel{\scriptscriptstyle{\rm def}}{{}={}}$}}

\newcommand {\half} {\mbox{$\frac{1}{2}$}}  %machine epsilon
\newcommand{\set}[2]{\left\{ #1 \;:\; #2 \right\}}

\newcommand {\macheps} {\mathbf{\epsilon}}  %machine epsilon
\newcommand {\real} {\mathbb{R}}
\newcommand {\nat} {\mathbb{N}}
\newcommand {\compl} {\mathbb{C}}
\newcommand {\diag} {{\mbox{diag}}}
\newcommand {\meas} {{\mbox{meas}}}
\newcommand {\mspan} {{\mbox{span}}}

\newcommand {\pdx} {\frac{\partial}{\partial x}}
\newcommand {\pdt} {\frac{\partial}{\partial t}}
\newcommand {\pdxx} {\frac{\partial^2}{\partial x^2}}
\newcommand {\dx} {\frac{d}{d x}}
\newcommand {\dt} {\frac{d}{d t}}
\newcommand {\dxx} {\frac{d^2}{d x^2}}

\newcommand {\bA} {\mbox{\boldmath $A$}}
\newcommand {\bB} {\mbox{\boldmath $B$}}
\newcommand {\bF} {\mbox{\boldmath $F$}}
\newcommand {\bI} {\mbox{\boldmath $I$}}
\newcommand {\bP} {\mbox{\boldmath $P$}}
\newcommand {\bY} {\mbox{\boldmath $Y$}}
\newcommand {\bZ} {\mbox{\boldmath $Z$}}
\newcommand {\ba} {\mbox{\boldmath $a$}}
\newcommand {\bb} {\mbox{\boldmath $b$}}
\newcommand {\bd} {\mbox{\boldmath $d$}}
\newcommand {\bff}{\mbox{\boldmath $f$}}
\newcommand {\bk} {\mbox{\boldmath $k$}}
\newcommand {\bp} {\mbox{\boldmath $p$}}
\newcommand {\bs} {\mbox{\boldmath $s$}}
\newcommand {\bu} {\mbox{\boldmath $u$}}
\newcommand {\bv} {\mbox{\boldmath $v$}}
\newcommand {\bw} {\mbox{\boldmath $w$}}
\newcommand {\bx} {\mbox{\boldmath $x$}}
\newcommand {\by} {\mbox{\boldmath $y$}}
\newcommand {\bz} {\mbox{\boldmath $z$}}
\newcommand {\blambda} {\mbox{\boldmath $\lambda$}}
\newcommand {\btau} {\mbox{\boldmath $\tau$}}

%
% JCM Quantum Chemistry commands
%
\newcommand{\bra}[1]{\langle #1|}
\newcommand{\ket}[1]{|#1\rangle}
\newcommand{\braket}[2]{\langle #1|#2\rangle}
% Example of usage
%\[
%\ket{\Psi}=\sum_{i}\ket{\phi_i}\braket{\phi_i}{\Psi}
%\]

%
% Other symbols
%
\newcommand {\grad}{\nabla}
\newcommand {\hess}{\nabla^2}
\newcommand {\condA}{\kappa(A)}

\newtheorem{algorithm}{Algorithm}

%

% \boxfig{pos}{wid}{text}:  A figure with a box around it
%
% pos	the usual figure placement arg: eg. htbp
% wid	the width of the figure, in some units: eg. 5in
% text	the contents of the figure, including picture/caption/label/etc
%
\newlength{\boxwidth}
\newcommand{\boxfigure}[3]{
	\begin{figure}[#1]
		\setlength{\boxwidth}{#2}
		\addtolength{\boxwidth}{.1in}

		\centering
		\framebox[\boxwidth]{
			\begin{minipage}{#2}
			#3
			\end{minipage}
		}
	\end{figure}  
}

% use \fullboxwidth for arg 2 of boxfigure to get box of size \textwidth


% \boxalg{pos}{text}:  An algorithm with a box around it
%
% title  the name of the algorithm
% text   the contents of the figure, including picture/caption/label/etc
%
\newcommand{\boxalg}[2]{
            \setlength{\boxwidth}{\fullboxwidth}
            \addtolength{\boxwidth}{.1in}
            \vspace{2ex}
            \framebox[\boxwidth][#1]{
                    \vspace{1ex}
                    \begin{minipage}{\fullinboxwidth}
                        #2
                    \end{minipage}
                    \vspace{1ex}
            }
            \vspace{2ex}
}






% see above
\newlength{\fullboxwidth}
\setlength{\fullboxwidth}{\textwidth}
\addtolength{\fullboxwidth}{-0.5in}

\newlength{\fullinboxwidth}
\setlength{\fullinboxwidth}{\fullboxwidth}
\addtolength{\fullinboxwidth}{-0.5in}




