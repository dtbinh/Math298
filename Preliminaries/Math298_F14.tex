\documentclass{article}

\usepackage{amsmath}
\usepackage{amssymb}
\usepackage{hyperref}
\usepackage{fullpage}
\usepackage[T1]{fontenc}
\usepackage{fourier}
\setlength{\parindent}{0cm}
% user defined macros

\newcommand{\secskip}{\vspace{6pt}}

\begin{document}
\centerline{{\bf \Large Fall 2014 Syllabus}}
\vspace{.2in}
{\bf \large Math 298: Directed Group Study} \\
{\bf \large Fundamental Concepts in Computational and Applied Mathematics} \\
\rule{6.5in}{.02in}

%\maketitle

\begin{tabbing}
{\bf Lecture time:} \quad \quad \= Monday, 1:30--2:20 pm \\
{\bf Lecture room:} \> SSM 150 \\
{\bf Instructor:} \> Juan C.~Meza \\
{\bf E-mail address:} \> {\tt jcmeza@ucmerced.edu} \\
{\bf Phone number:} \> (209) 228-4487 \\
{\bf Office:} \> COB 360 \\
{\bf Office hours:} \> M, 2:30-3:30 PM and by appointment. \\
\end{tabbing}

\section*{Course goal and topics} Starting research in applied mathematics can be like learning a new language.  There are many new terms and basic facts that must be mastered in order to even start a conversation about one's research.  In addition, there are many expectations and new skills beyond those taught in standard undergraduate mathematics courses that must be learned to become a successful graduate student in applied mathematics.  This course will introduce the student to some of the fundamental concepts used in computational and applied mathematics.  We will not attempt to go into any one area in depth; rather we will present a survey of some of the key ideas and tricks used by practicing computational mathematicians.  Along the way, our tour will highlight some of the classic numerical analysis papers and the top algorithms used today including those from linear algebra, nonlinear equations, optimization, discretizations and differential equations, and spectral methods.  These ideas will also be highlighted through several case studies taken from real-world applications in computational sciences.

\secskip

\noindent
\section*{Learning outcomes} Upon completion of this course, students should:
\begin{enumerate}

\item Be familiar with key mathematical concepts used in developing numerical algorithms.
 
\item Understand some of the basic skills and resources necessary to start research in computational and applied mathematics.

\item Be aware of basic communications skills needed to present mathematics clearly to a
  broad audience in writing and in speaking.
  
\end{enumerate}

\secskip

\noindent
Math 298: Directed Group Study addresses the following two Program Learning Outcomes of the Applied Mathematics
Ph.D.~and M.S.~programs:

\begin{description}
\item[\quad PLO \#3:] Give clear and organized written and verbal explanations
  of mathematical ideas to a variety of audiences including teaching
  undergraduate students.
\item[\quad PLO \#4:] Model real-world problems mathematically and analyze those models using their mastery of the core concepts. 
\end{description}

\section*{Directed group study meetings}

The main vehicle for learning the concepts in this course will be through the class discussions.  This is in keeping with the idea that in order to learn a new language, one must practice speaking the language.   As such, the class presentations will have a strong element of participation by everybody in the group.  It will be expected that all participants will come to class ready to discuss the topic of the week and to present material as needed.  As the semester progresses, we will also work on basic communication skills for presentations.  The students will then be asked to choose a case-study from computational science or mathematics and lead a discussion during one of the meetings. 

\section*{Course materials}

\begin{description}

\item {\bf Textbook.} No text or other materials are required for this course.  We will make frequent references to some of the major applied mathematics and numerical analysis papers in the literature most of which can be found on the web (see e.g. \url{http://people.maths.ox.ac.uk/trefethen/classic_papers.txt}).  Some familiarity with these papers and their concepts will be useful to the student, but is not required prior to the course.  In addition, having access to one or two good numerical analysis books throughout the course will help students in their appreciation of the techniques we will discuss.  References to all relevant materials will be provided as needed.

\item {\bf Course webpage.} The Math 298 website is part of the
  UCMCROPS course management system. All important course materials
  will be posted under RESOURCES on this website.

\end{description}

\section*{Grade determination} 

\begin{description}

\item Your grade in this class will be determined by the following combination:

\begin{itemize}
\item[50\%]  Class participation
\item[25\%]  Reading and writing assignments
\item[25\%]  Presentations at end of the semester
\end{itemize}

\end{description}

\section*{Additional course information}

\begin{description}

\item \textbf{Dropping the course.} You may drop this course without
  paying a fee and without further approval before 5:00 pm on
  Thursday, September 19.  Dropping the course after this time
  requires the signed approval of the instructor, and the confirmation
  of the Dean of the School of Natural Sciences. Please see the UC
  Merced \textit{General Catalog} for more details.

\secskip

\item \textbf{Special accommodations.} If you qualify for
  accommodations because of a disability, please submit a letter from
  Disability Services to the instructor in a timely manner so that
  your needs may be addressed.  Student Affairs determines
  accommodations based on documented disabilities.

  We will make every effort to accommodate all students who, because
  of religious obligations, have conflicts with scheduled exams,
  assignments, or required attendance.  Please speak with me during
  the first week of classes regarding any potential academic
  adjustments or accommodations that may arise due to religious
  beliefs during this term.

\secskip

\item \textbf{Academic integrity.}  Academic integrity is the
  foundation of an academic community and without it none of the
  educational or research goals of the university can be achieved.
  All members of the university community are responsible for its
  academic integrity.  Existing policies forbid cheating on
  examinations, plagiarism and other forms of academic dishonesty.
  The UC Merced Academic Honesty Policy and Adjudication Procedures
  available on the website: \url{http://studentlife.ucmerced.edu} by
  following the link to Student Judicial Affairs.

\end{description}

\clearpage

\begin{center}

{\bf Math 298 Tentative Class Schedule}

\bigskip

\begin{tabular}{|l|l|l|l|}

  \hline

  {\bf Week} & {\bf Date} & {\bf Topic}\\

  \hline \hline

  Week 1 & Sept.~2 (Labor Day)  & No Class \\
  \hline

  Week 2 & Sept.~9  & Introduction to Numerical Analysis Motifs\\
  \hline

  Week 3 & Sept.~16  & Dense Linear Algebra -- Cornerstone of Comp. Math \\
  \hline

  Week 4 & Sept.~23  & Sparse Linear Algebra -- Taking Advantage of Structure \\
  \hline

  Week 5 & Sept.~30  & Structured Grids -- Beginning Discretizations \\
  \hline

  Week 6 & Oct.~7    & Unstructured Grids -- Complexity of Real World Problems \\
  \hline

  Week 7 & Oct.~14   & Spectral Methods  -- Transforming the Problem \\
  \hline

  Week 8 & Oct.~21   & N-Body problems -- When is Near Near Enough? \\
  \hline

  Week 9 & Oct.~28   & Nonlinear Equations and Optimization -- Making Optimal Decisions \\
  \hline

  Week 10 & Nov.~4    & Presentation Essentials \\
  \hline

  Week 11 & Nov.~11 (Veterans Day) & No class \\
  \hline

  Week 12 & Nov.~18   & Student Presentations\\
  \hline

  Week 13 & Nov.~25   & Student Presentations\\
  \hline

  Week 14 & Dec.~2    & Student Presentations\\
  \hline

  Week 15 & Dec.~9    & Student Presentations\\
  \hline
  
\end{tabular}

\end{center}

\

\end{document}
