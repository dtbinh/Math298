\documentclass{article}

\usepackage{amsmath}
\usepackage{amssymb}
\usepackage{fullpage}
\usepackage[T1]{fontenc}
\usepackage{fourier}

% user defined macros

\newcommand{\secskip}{\vspace{6pt}}

\begin{document}

\title{{\bf Math 298: Directed Group Study}}
\author{{\bf 1 Unit}}
\date{{\bf Fall 2012 Course Syllabus}}
\maketitle

\section*{Course goals, topics, and learning outcomes}

{\bf Course goal.} To understand the expectations for graduate studies
in applied mathematics and to develop basic skills needed for applied
mathematics research.

\secskip

{\bf Learning outcomes.} Upon completion of this course, students
should be able to 
\begin{enumerate}

\item Know the expectations of graduate students in the applied
  mathematics graduate studies including required courses, and
  preliminary exams, for example.

\item Perform basic computational tasks on servers such as login
  remotely; upload/download data; manage files; use an editor such as
  vim or emacs; write, compile and execute basic programs in C and
  Fortran; write and run Matlab scripts.

\item Begin to engage with the applied mathematics literature and
  learn the tools available through our library that are needed to
  find and cite references.

\item Prepare documents using \LaTeX, and begin to practice
  communicating mathematics and mathematical results clearly to a
  broad audience in writing and in speaking.

\item Know what skills are required to participate actively in applied
  and computational mathematics research, and what opportunities are
  available.

\end{enumerate}

\secskip

{\bf Relationship to Program Learning Outcomes.} Math 298: Directed
Group Study addresses two of the five Program Learning Outcomes of the
Ph.D.~and M.S.~programs listed below.
\begin{enumerate}

\item PLO \#3: Give clear and organized written and verbal explanations
  of mathematical ideas to a variety of audiences including teaching
  undergraduate students.

\item PLO \#5: Recognize ethical and responsible conduct and learn how
  to apply them to research.

\end{enumerate}

\section*{Directed group study meetings}

\begin{description}

\item {\bf Purpose.} We will introduce new concepts, run through
  demonstrations, and hold discussions.

\item {\bf Lecture time.} M, 1:30 pm -- 2:20 pm

\item {\bf Lecture room.} CLSSRM 272

\item {\bf Instructor.} Arnold D.~Kim

  \begin{description}

  \item {\bf E-mail address.} {\tt adkim@ucmerced.edu}

  \item {\bf Phone number.} (209) 228.2951

  \item {\bf Office.} CLSSRM 368

  \item {\bf Office hours.} MW, 3:30 pm - 4:30 pm, or by appointment.

  \end{description}

\end{description}

\section*{Course materials}

\begin{description}

\item {\bf Textbook.} No text or other materials are required for this
  course.

\item {\bf Course webpage.} The Math 298 website is part of the
  UCMCROPS course management system. All important course materials
  will be posted under RESOURCES on this website.

\item {\bf Library.} Students will learn to engage with the applied
  mathematics literature through the online tools provided by our
  library including MathSciNet, Inspec, RefWorks, various online
  journals, and interlibrary loan to name a few.

\item {\bf Course wiki.} As part of this course, students will develop
  a wiki page and contribute content to it that will serve to provide
  information to all applied math students and faculty on basic skills
  needed for research.

\end{description}

\section*{Grade determination} 

\begin{description}

\item Your grade in this class is determined entirely by class
  participation and homework assignments.

\end{description}

\section*{Additional course information}

\begin{description}

\item \textbf{Dropping the course.} You may drop this course without
  paying a fee and without further approval before 5:00 pm on
  Thursday, September 15.  Dropping the course after this time
  requires the signed approval of the instructor, and the confirmation
  of the Dean of the School of Natural Sciences. Please see the UC
  Merced \textit{General Catalog} for more details.

\secskip

\item \textbf{Special accommodations.} If you qualify for
  accommodations because of a disability, please submit a letter from
  Disability Services to the instructor in a timely manner so that
  your needs may be addressed.  Student Affairs determines
  accommodations based on documented disabilities.

  We will make every effort to accommodate all students who, because
  of religious obligations, have conflicts with scheduled exams,
  assignments, or required attendance.  Please speak with me during
  the first week of classes regarding any potential academic
  adjustments or accommodations that may arise due to religious
  beliefs during this term.

\secskip

\item \textbf{Academic integrity.}  Academic integrity is the
  foundation of an academic community and without it none of the
  educational or research goals of the university can be achieved.
  All members of the university community are responsible for its
  academic integrity.  Existing policies forbid cheating on
  examinations, plagiarism and other forms of academic dishonesty.
  The UC Merced Academic Honesty Policy and Adjudication Procedures
  available on the website: http://studentlife.ucmerced.edu by
  following the link to Student Judicial Affairs.

\end{description}

\clearpage

\begin{center}

{\bf Math 298 Tentative Class Schedule}

\bigskip

\begin{tabular}{|l|l|l|l|}

  \hline

  {\bf Week} & {\bf Date} & {\bf Group Meeting} & {\bf Topic}\\

  \hline \hline

  Week 1 & Aug.~27  & Group Meeting 1  & Introductions and Syllabus\\
  \hline
  Week 2 & Sept.~3  & Labor Day  & No Class \\
  \hline

  Week 3 & Sept.~10 & Group Meeting 2  & Requirements and expectations
  of graduate students\\
  \hline

  Week 4 & Sept.~17 & Group Meeting 3 & Discussion on topics covered
  on the preliminary exams\\
  \hline

  Week 5 & Sept.~24 & Group Meeting 4 & Server Computing: ssh,
  X-windows, file managment, text editors\\
  \hline

  Week 6 & Oct.~1   & Group Meeting 5 & Programming basics in C and
  Fortran\\
  \hline

  Week 7 & Oct.~8   & Group Meeting 6 & Applied
  mathematics literature and our library\\
  \hline

  Week 8 & Oct.~15  & Group Meeting 7 & Basics of writing a
  mathematics document using \LaTeX\\
  \hline

  Week 9 & Oct.~22  & Group Meeting 8 & Using \LaTeX to prepare
  slides and posters\\
  \hline

  Week 10 & Oct.~29  & Group Meeting 9 & Writing and presenting
  mathematical results\\
  \hline

  Week 11 & Oct.~5   & Group Meeting 10 & Scholarly research and
  ethics\\
  \hline

  Week 12 & Nov.~12  & Veterans Day & No class \\
  \hline

  Week 13 & Nov.~19  & Group Meeting 11 & Applied Math Faculty
  Research Presentations\\
  \hline

  Week 14 & Nov.~26  & Group Meeting 12 & Applied Math Faculty
  Research Presentations\\
  \hline

  Week 15 & Dec.~3   & Group Meeting 13 & Applied Math Faculty
  Research Presentations\\

  \hline

\end{tabular}

\end{center}


\end{document}
