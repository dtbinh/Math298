\documentclass[mathserif,notes]{beamer}

\usepackage{pdfsync} % works only on TeXShop, comment otherwise
\usepackage{pgf}
\usepackage{multimedia} % to embed movies in the PDF file
\usepackage{xcolor}
\usepackage{graphicx}
\usepackage{comment}
\usepackage{pgfpages}
\usepackage{topcapt}
\usepackage{booktabs}

% for the tikz pictures
\usepackage{geometry}
%\geometry{hmargin=1cm,vmargin=1cm}
\usepackage{tikz}
\def\width{5}
\def\hauteur{5}
\mode<presentation>
{
  \usetheme{Madrid}   %  \usetheme{Warsaw}
  % Note: this next command makes nice slides, but Adobe doesn't like it (you get blank pages)
  \setbeamertemplate{background canvas}[vertical shading][bottom=white,top=structure.fg!15]
 % \setbeamertemplate{footline}{\pgfuseimage{UCM-logo}}

  %\usecolortheme{seahorse}
  \usecolortheme{beaver}
 % \setbeamercovered{transparent}
}
%\usepackage[T1]{fontenc}
% For printing purposes only
%\pgfpagesuselayout{2 on 1}[letterpaper,border shrink=5mm,portrait]
\pgfpagesuselayout{resize to}[letterpaper,border shrink=5mm,landscape]  % print out full size on paper

\usepackage{amssymb,amsmath}
\hypersetup{%
 pdftitle={Math298 Lecture 7},
 pdfauthor={Juan Meza(UC Merced)},
 pdfsubject={Fundamental Concepts in Computational and Applied Mathematics},
 pdfkeywords={N-Body, Fast Multipole}
 }
 
\title[Math 298 Lecture 7: N-Body]{Fundamental Concepts in \\ Computational and Applied Mathematics}  \subtitle{}
\author[Juan Meza]{Juan Meza \\ School of Natural Sciences \\ University of California, Merced}
\date[Fall 2014]{Fall 2014}
\institute[UC Merced]

%\pgfdeclareimage[height=0.35cm]{UCM-logo}{UCM-logo}
% \logo{\pgfuseimage{UCM-logo}}
\pgfdeclareimage[width=0.7\textwidth]{galaxy}{galaxy}
\pgfdeclareimage[width=0.6\textwidth]{FMM.key}{FMM.key}
%% figures

% My LaTeX definitions with some common abbreviations





% Common abbreviations
% (Remember to put '\ ' after if an interword space is
%  desired rather than end-of-sentence space. Same for '.etc)' ).
\newcommand{\eg}{{\em e.g.}}		% e.g.
\newcommand{\ie}{{\em i.e.}}		% i.e.
\newcommand{\etc}{{\em etc.}}		% etc.
\newcommand{\vs}{{\em vs.}}		% vs.
\newcommand{\remark}{{\color{red} Remark:} }


\newcommand {\DS} {\displaystyle}
\newcommand{\eref}[1]{{\rm{(\ref{#1})}}}
\newcommand{\tref}[1]{{\rm{\ref{#1}}}}


\def\frechet{Fr\'echet\ }
\def\more{Mor\'e\ }
\def\fl{{\sf fl}}
\def\flop{{\sf flop} }
\def\flops{{\sf flops} }
\def\matlab{{\sc Matlab} }
\def\Matlab{\matlab}

\def\bigO{\mathcal{O}}

% Mathematical Symbols

\newcommand{\deq}{\raisebox{0pt}[1ex][0pt]{$\stackrel{\scriptscriptstyle{\rm def}}{{}={}}$}}

\newcommand {\half} {\mbox{$\frac{1}{2}$}}  %machine epsilon
\newcommand{\set}[2]{\left\{ #1 \;:\; #2 \right\}}

\newcommand {\macheps} {\mathbf{\epsilon}}  %machine epsilon
\newcommand {\real} {\mathbb{R}}
\newcommand {\nat} {\mathbb{N}}
\newcommand {\compl} {\mathbb{C}}
\newcommand {\diag} {{\mbox{diag}}}
\newcommand {\meas} {{\mbox{meas}}}
\newcommand {\mspan} {{\mbox{span}}}

\newcommand {\pdx} {\frac{\partial}{\partial x}}
\newcommand {\pdt} {\frac{\partial}{\partial t}}
\newcommand {\pdxx} {\frac{\partial^2}{\partial x^2}}
\newcommand {\dx} {\frac{d}{d x}}
\newcommand {\dt} {\frac{d}{d t}}
\newcommand {\dxx} {\frac{d^2}{d x^2}}

\newcommand {\bA} {\mbox{\boldmath $A$}}
\newcommand {\bB} {\mbox{\boldmath $B$}}
\newcommand {\bF} {\mbox{\boldmath $F$}}
\newcommand {\bI} {\mbox{\boldmath $I$}}
\newcommand {\bP} {\mbox{\boldmath $P$}}
\newcommand {\bY} {\mbox{\boldmath $Y$}}
\newcommand {\bZ} {\mbox{\boldmath $Z$}}
\newcommand {\ba} {\mbox{\boldmath $a$}}
\newcommand {\bb} {\mbox{\boldmath $b$}}
\newcommand {\bd} {\mbox{\boldmath $d$}}
\newcommand {\bff}{\mbox{\boldmath $f$}}
\newcommand {\bk} {\mbox{\boldmath $k$}}
\newcommand {\bp} {\mbox{\boldmath $p$}}
\newcommand {\bs} {\mbox{\boldmath $s$}}
\newcommand {\bu} {\mbox{\boldmath $u$}}
\newcommand {\bv} {\mbox{\boldmath $v$}}
\newcommand {\bw} {\mbox{\boldmath $w$}}
\newcommand {\bx} {\mbox{\boldmath $x$}}
\newcommand {\by} {\mbox{\boldmath $y$}}
\newcommand {\bz} {\mbox{\boldmath $z$}}
\newcommand {\blambda} {\mbox{\boldmath $\lambda$}}
\newcommand {\btau} {\mbox{\boldmath $\tau$}}

%
% JCM Quantum Chemistry commands
%
\newcommand{\bra}[1]{\langle #1|}
\newcommand{\ket}[1]{|#1\rangle}
\newcommand{\braket}[2]{\langle #1|#2\rangle}
% Example of usage
%\[
%\ket{\Psi}=\sum_{i}\ket{\phi_i}\braket{\phi_i}{\Psi}
%\]

%
% Other symbols
%
\newcommand {\grad}{\nabla}
\newcommand {\hess}{\nabla^2}
\newcommand {\condA}{\kappa(A)}

\newtheorem{algorithm}{Algorithm}

%

% \boxfig{pos}{wid}{text}:  A figure with a box around it
%
% pos	the usual figure placement arg: eg. htbp
% wid	the width of the figure, in some units: eg. 5in
% text	the contents of the figure, including picture/caption/label/etc
%
\newlength{\boxwidth}
\newcommand{\boxfigure}[3]{
	\begin{figure}[#1]
		\setlength{\boxwidth}{#2}
		\addtolength{\boxwidth}{.1in}

		\centering
		\framebox[\boxwidth]{
			\begin{minipage}{#2}
			#3
			\end{minipage}
		}
	\end{figure}  
}

% use \fullboxwidth for arg 2 of boxfigure to get box of size \textwidth


% \boxalg{pos}{text}:  An algorithm with a box around it
%
% title  the name of the algorithm
% text   the contents of the figure, including picture/caption/label/etc
%
\newcommand{\boxalg}[2]{
            \setlength{\boxwidth}{\fullboxwidth}
            \addtolength{\boxwidth}{.1in}
            \vspace{2ex}
            \framebox[\boxwidth][#1]{
                    \vspace{1ex}
                    \begin{minipage}{\fullinboxwidth}
                        #2
                    \end{minipage}
                    \vspace{1ex}
            }
            \vspace{2ex}
}






% see above
\newlength{\fullboxwidth}
\setlength{\fullboxwidth}{\textwidth}
\addtolength{\fullboxwidth}{-0.5in}

\newlength{\fullinboxwidth}
\setlength{\fullinboxwidth}{\fullboxwidth}
\addtolength{\fullinboxwidth}{-0.5in}






\definecolor{navy}{RGB}{0,0,128}
\definecolor{forestgreen}{RGB}{34,139,34}
\definecolor{mylightsteelblue}{RGB}{176,196,222}
\definecolor{mylightcyan}{RGB}{180,255,255}
\definecolor{mypaleturquoise}{RGB}{175,238,238}
\definecolor{mylightgoldenrod}{RGB}{250,250,210}
\definecolor{mylightyellow}{RGB}{255,255,224}
\definecolor{mylightsalmon}{RGB}{255,160,122}

\setbeamercolor{blacklightsteelblue}{fg=black,bg=mylightsteelblue}
\setbeamercolor{blackpaleturquoise}{fg=black,bg=mypaleturquoise}
\setbeamercolor{blacklightcyan}{fg=black,bg=mylightcyan}
\setbeamercolor{blacklightgoldenrod}{fg=black,bg=mylightgoldenrod}
\setbeamercolor{blacklightyellow}{fg=black,bg=mylightyellow}
\setbeamercolor{blacklightsalmon}{fg=black,bg=mylightsalmon}
\setbeamercolor{blackcyan}{fg=black,bg=cyan}


\begin{document}

%%%%%%%%%%%%%%%%%%%%%%%%%%%%%%%%%%%%%%%%%%%%%
\frame{

\titlepage

} % end frame

\section{Introduction}
%%%%%%%%%%%%%%%%%%%%%%%%%%%%%%%%%%%%%%%%%%%%%
\frame{\frametitle{N-Body methods}

\begin{columns}
\column{.5\textwidth}
\begin{itemize}
\item Involve computation of interactions between N-bodies/particles
\item Examples arise in molecular dynamics, gravitation, electrostatics, etc.
\item Also useful for solution of boundary value problems,  biharmonic equations, Poisson equation, etc.
\end{itemize}
\column{.5\textwidth}
	\pgfputat{\pgfxy(0,-4.0)}{\pgfbox[left,base]{\pgfuseimage{galaxy}}}
\end{columns}
} % end frame

%%%%%%%%%%%%%%%%%%%%%%%%%%%%%%%%%%%%%%%%%%%%%
\frame{\frametitle{N-Body Problem Description}

Suppose we are given a set of  {\color{navy} source} points $x_i, i=1, \ldots, N$ and we want to compute sums of the following form:
\begin{equation}
u(x_i) = \sum^N_{j=1} w_j K(x,y_j), \label{eq:nbody}
\end{equation}
where
\begin{itemize}

\item $y_j, j=1, \ldots, N$ are called the {\color{navy} target} points
\item $w_j$ are source weights
\item $K(x,y_j)$ is called the kernel, e.g. potential function
\end{itemize}
\vspace{.25in}
\remark: A straightforward algorithm would appear to be $\bigO (N^2)$ 
} % end frame


\section{Examples}
%%%%%%%%%%%%%%%%%%%%%%%%%%%%%%%%%%%%%%%%%%%%%
\frame{\frametitle{Example: N-Body problem of gravitation}

The gravitational potential is given by 
$$
\Phi(x_j) = \sum^{N}_{\substack{
i=1 \\
i \neq j}} \frac{m_i}{r_{ij}}, \quad r_{ij} = || x_i - x_j ||,
$$
and the gravitational field $E$ by:
$$
E(x_j) = \sum^{N}_{\substack{
i=1 \\
i \neq j}} m_i \frac{x_j - x_i}{r^{3}_{ij}}
$$

\remark Same equations applicable to electrostatics
} % end frame


%%%%%%%%%%%%%%%%%%%%%%%%%%%%%%%%%%%%%%%%%%%%%
\frame{\frametitle{N-Body Problem (Short Detour)}

Recall one such problem from our previous class, i.e. the FFT:
$$
u_j = \sum^N_{k=1} e^{2\pi i j k / N} \cdot w_k , \qquad j = 1, \ldots, N.
$$

\begin{block}{Question:}
What is the complexity for such an algorithm?
\end{block} % Answer: O(NlogN)
} % end frame

%%%%%%%%%%%%%%%%%%%%%%%%%%%%%%%%%%%%%%%%%%%%%
\frame{\frametitle{Finite Rank/Degenerate Kernels}

First consider a kernel which can be written as:
$$
K(x,y) =  \sum^p_{k=1} \phi_k(x) \psi_k(y). 
$$
These are called {\color{forestgreen} finite rank or degenerate} kernels.  \\
\vspace{.25in}
We can reduce our original problem Eq. (\ref{eq:nbody}) to the following 2-step procedure.
\begin{enumerate}
\item Compute $A_k = \sum^N_{i=1} w_i \psi_k (y)$
\item Evaluate $u(x) = \sum^p_{k=1} A_k \phi_k (x)$
\end{enumerate}
\begin{block}{Question:}
What is the complexity for such an algorithm?
\end{block} % Answer: O(Np)
} % end frame

%%%%%%%%%%%%%%%%%%%%%%%%%%%%%%%%%%%%%%%%%%%%%
\frame{\frametitle{Discussion}

\begin{block}{Question}

Can you think of another example where you might be able to use this idea?

\end{block}

} % end frame


%%%%%%%%%%%%%%%%%%%%%%%%%%%%%%%%%%%%%%%%%%%%%
\frame{\frametitle{Motivation}

Like many other problems first take a look at the structure of the problem
\begin{itemize}
\item Forces can usually be broken down into ``short-range" and ``long-range"
\item Can we take advantage of this to develop faster algorithms?
\end{itemize}
\vspace{.25in}
2 Key Ideas
\begin{enumerate}
\item Replace group of ``distant'' particles by one ``pseudo-particle"
\item Decompose space into a hierarchy of areas that are suitably ``distant"
\end{enumerate}

} % end frame

%%%%%%%%%%%%%%%%%%%%%%%%%%%%%%%%%%%%%%%%%%%%%
\frame{\frametitle{Pictorially}

\begin{columns}
\column{.4\textwidth}
\begin{itemize}
\item Brute force method would compute the interactions between each pair of particles
\item Idea is to combine the effects of several nearby particles into one 
\item Then the interaction can be reduced to just 2 "particles"
\end{itemize}
\column{.6\textwidth}
	\pgfputat{\pgfxy(0,-3)}{\pgfbox[left,base]{\pgfuseimage{FMM.key}}}
\end{columns}
} % end frame

%%%%%%%%%%%%%%%%%%%%%%%%%%%%%%%%%%%%%%%%%%%%%
\frame{\frametitle{Replacing group of particles: multipole expansion}

%A multipole expansion is a series expansion of the effect produced by a given system in terms of an expansion parameter which becomes small as the distance away from the system increases. Therefore, the leading one or terms in a multipole expansion are generally the strongest. The first-order behavior of the system at large distances can therefore be obtained from the first terms of this series, which is generally much easier to compute than the general solution. Multipole expansions are most commonly used in problems involving the gravitational field of mass aggregations, the electric and magnetic fields of charge and current distributions, and the propagation of electromagnetic waves. Wolfram, http://scienceworld.wolfram.com/physics/MultipoleExpansion.html

Example: Electrostatic Potential due to a set of charges $q_i$ located at $x_i$\\
\vspace{.15in}
Want $$ K(y-x) = \frac{1}{|y - x|} \approx \sum_{k=0}^{p} \phi_k(x) \psi_k(y),$$
which can be written as
$$
 \frac{1}{|y - x|} = \frac{1}{|y|} \sum_{n=0}^{\infty} P_n(\cos \theta) \left ( \frac{|x|}{|y|} \right )^n ,
$$
where $P_n(\cos \theta)$ are the Legendre polynomials.\\
\vspace{.15in}
\begin{block}{N.B.} 
Series is convergent for $v = \frac{|x|}{|y|} < 1 $ 
\end{block}
} % end frame

%%%%%%%%%%%%%%%%%%%%%%%%%%%%%%%%%%%%%%%%%%%%%
\frame{\frametitle{Hierarchy of domains}

\begin{itemize}
\item Similar to our old friend divide and conquer
\item Need to be careful about dividing space
\item Too coarse a division and your approximation is not good enough
\item Too fine a division leads you back to the original problem
\end{itemize}

}% end frame


%%%%%%%%%%%%%%%%%%%%%%%%%%%%%%%%%%%%%%%%%%%%%
\frame{\frametitle{Four key features of an FMM code \footnote{Beatson and Greengard\cite{BeGr09}} }

\begin{itemize}
\item A specified acceptable accuracy
\item A hierarchical subdivision of space into panels or clusters of sources
\item A far field expansion of the kernel in which the influence of source and evaluation of points separates
\item (Optional) Conversion of far field expansions into local expansions
\end{itemize}
}% end frame



\section{Summary}
%%%%%%%%%%%%%%%%%%%%%%%%%%%%%%%%%%%%%%%%%%%%%
\frame{\frametitle{Comparison of FFT and FMM}
% Requires the booktabs if the memoir class is not being used
\begin{table}[htbp]
   \centering
   \topcaption{Comparison of FFT with FMM} % requires the topcapt package
   \begin{tabular}{@{} lcr @{}} % Column formatting, @{} suppresses leading/trailing space
      \toprule
%      \cmidrule(r){1-2} % Partial rule. (r) trims the line a little bit on the right; (l) & (lr) also possible
      Property   & FFT & FMM \\
      \midrule
      Work         & $5N \log N$             & $N \log N$ \\
      Accuracy   & exact                       & approximate \\
      Domain     & uniform spatial grid  & any \\
      Based on  & Algebra                   & Analytics \\
      \bottomrule
   \end{tabular}
   \label{tab:booktabs}
\end{table}

}% end frame


%%%%%%%%%%%%%%%%%%%%%%%%%%%%%%%%%%%%%%%%%%%%%
\begin{frame}[allowframebreaks]
  \frametitle<presentation>{References}    
  \begin{thebibliography}{10}    

 \beamertemplatebookbibitems
 \bibitem{BeGr09}
 R. Beatson and L. Greengard
    \newblock {\em A Short Course on Fast Multipole Methods},
   \newblock Cambridge University Press, 2009.

 \beamertemplatearticlebibitems
 \bibitem{GrRo87}
L. Greengard and V. Rokhlin.
  \newblock A Fast Algorithm for Particle Simulations,
 \newblock J. Comp. Phys. 73, 325-348, 1987.
 
\beamertemplateonlinebibitems
 \bibitem{Ch}
 Vikas Chandrakant Raykar
    \newblock {\em A short primer on the fast multipole method},
   \newblock \url{http://www.umiacs.umd.edu/labs/cvl/pirl/vikas/publications/FMM_tutorial.pdf}

 \end{thebibliography}
 \end{frame}
 
\end{document}
