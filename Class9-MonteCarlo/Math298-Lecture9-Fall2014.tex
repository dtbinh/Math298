\documentclass[mathserif,notes]{beamer}

\usepackage{pdfsync} % works only on TeXShop, comment otherwise
\usepackage{pgf}
\usepackage{multimedia} % to embed movies in the PDF file
\usepackage{xcolor}
\usepackage{graphicx}
\usepackage{comment}
\usepackage{pgfpages}
\usepackage{topcapt}
\usepackage{booktabs}
%
% for algorithm
\usepackage{algorithm2e}
\usepackage{algorithmic}
\usepackage{float}

% for the tikz pictures
\usepackage{geometry}
%\geometry{hmargin=1cm,vmargin=1cm}
\usepackage{tikz}
\def\width{5}
\def\hauteur{5}
\mode<presentation>
{
  \usetheme{Madrid}   %  \usetheme{Warsaw}
  % Note: this next command makes nice slides, but Adobe doesn't like it (you get blank pages)
  \setbeamertemplate{background canvas}[vertical shading][bottom=white,top=structure.fg!15]
  %\setbeamertemplate{footline}{\pgfuseimage{UCM-logo}}

  %\usecolortheme{seahorse}
  \usecolortheme{beaver}
 % \setbeamercovered{transparent}
}
%\usepackage[T1]{fontenc}
% For printing purposes only
%\pgfpagesuselayout{2 on 1}[letterpaper,border shrink=5mm,portrait]
\pgfpagesuselayout{resize to}[letterpaper,border shrink=5mm,landscape]  % print out full size on paper

\usepackage{amssymb,amsmath}
\hypersetup{%
 pdftitle={Math298 Lecture 8},
 pdfauthor={Juan Meza(UC Merced)},
 pdfsubject={Fundamental Concepts in Computational and Applied Mathematics},
 pdfkeywords={Monte Carlo, Markov Chain, MCMC}
 }
 
\title[Math 298 Lecture 9: Monte Carlo]{Fundamental Concepts in \\ Computational and Applied Mathematics}  \subtitle{}
\author[Juan Meza]{Juan Meza \\ School of Natural Sciences \\ University of California, Merced}
\date[Fall 2014]{Fall 2014}
\institute[UC Merced]

% My LaTeX definitions with some common abbreviations
%




% Common abbreviations
% (Remember to put '\ ' after if an interword space is
%  desired rather than end-of-sentence space. Same for '.etc)' ).
\newcommand{\eg}{{\em e.g.}}		% e.g.
\newcommand{\ie}{{\em i.e.}}		% i.e.
\newcommand{\etc}{{\em etc.}}		% etc.
\newcommand{\vs}{{\em vs.}}		% vs.



\newcommand {\DS} {\displaystyle}
\newcommand{\eref}[1]{{\rm{(\ref{#1})}}}
\newcommand{\tref}[1]{{\rm{\ref{#1}}}}


\def\frechet{Fr\'echet\ }
\def\more{Mor\'e\ }
\def\fl{{\sf fl}}
\def\flop{{\sf flop} }
\def\flops{{\sf flops} }
\def\matlab{{\sc Matlab} }
\def\Matlab{\matlab}



% Mathematical Symbols

\newcommand{\deq}{\raisebox{0pt}[1ex][0pt]{$\stackrel{\scriptscriptstyle{\rm def}}{{}={}}$}}


\newcommand {\half} {\mbox{$\frac{1}{2}$}}  %machine epsilon
\newcommand{\set}[2]{\left\{ #1 \;:\; #2 \right\}}


\newcommand {\macheps} {\mathbf{\epsilon}}  %machine epsilon
\newcommand {\real} {\mathbb{R}}
\newcommand {\nat} {\mathbb{N}}
\newcommand {\compl} {\mathbb{C}}
\newcommand {\diag} {{\mbox{diag}}}
\newcommand {\meas} {{\mbox{meas}}}
\newcommand {\mspan} {{\mbox{span}}}

\newcommand {\pdx} {\frac{\partial}{\partial x}}
\newcommand {\pdt} {\frac{\partial}{\partial t}}
\newcommand {\pdxx} {\frac{\partial^2}{\partial x^2}}
\newcommand {\dx} {\frac{d}{d x}}
\newcommand {\dt} {\frac{d}{d t}}
\newcommand {\dxx} {\frac{d^2}{d x^2}}

\newcommand {\bA} {\mbox{\boldmath $A$}}
\newcommand {\bB} {\mbox{\boldmath $B$}}
\newcommand {\bF} {\mbox{\boldmath $F$}}
\newcommand {\bI} {\mbox{\boldmath $I$}}
\newcommand {\bP} {\mbox{\boldmath $P$}}
\newcommand {\bY} {\mbox{\boldmath $Y$}}
\newcommand {\bZ} {\mbox{\boldmath $Z$}}
\newcommand {\ba} {\mbox{\boldmath $a$}}
\newcommand {\bb} {\mbox{\boldmath $b$}}
\newcommand {\bd} {\mbox{\boldmath $d$}}
\newcommand {\bff}{\mbox{\boldmath $f$}}
\newcommand {\bk} {\mbox{\boldmath $k$}}
\newcommand {\bp} {\mbox{\boldmath $p$}}
\newcommand {\bs} {\mbox{\boldmath $s$}}
\newcommand {\bu} {\mbox{\boldmath $u$}}
\newcommand {\bv} {\mbox{\boldmath $v$}}
\newcommand {\bw} {\mbox{\boldmath $w$}}
\newcommand {\bx} {\mbox{\boldmath $x$}}
\newcommand {\by} {\mbox{\boldmath $y$}}
\newcommand {\bz} {\mbox{\boldmath $z$}}
\newcommand {\blambda} {\mbox{\boldmath $\lambda$}}
\newcommand {\btau} {\mbox{\boldmath $\tau$}}

%
% JCM Quantum Chemistry commands
%
\newcommand{\bra}[1]{\langle #1|}
\newcommand{\ket}[1]{|#1\rangle}
\newcommand{\braket}[2]{\langle #1|#2\rangle}
% Example of usage
%\[
%\ket{\Psi}=\sum_{i}\ket{\phi_i}\braket{\phi_i}{\Psi}
%\]

%
% Other symbols
%
\newcommand {\grad}{\nabla}
\newcommand {\hess}{\nabla^2}
\newcommand {\condA}{\kappa(A)}


\newtheorem{algorithm}{Algorithm}

%

% \boxfig{pos}{wid}{text}:  A figure with a box around it
%
% pos	the usual figure placement arg: eg. htbp
% wid	the width of the figure, in some units: eg. 5in
% text	the contents of the figure, including picture/caption/label/etc
%
\newlength{\boxwidth}
\newcommand{\boxfigure}[3]{
	\begin{figure}[#1]
		\setlength{\boxwidth}{#2}
		\addtolength{\boxwidth}{.1in}

		\centering
		\framebox[\boxwidth]{
			\begin{minipage}{#2}
			#3
			\end{minipage}
		}
	\end{figure}  
}

% use \fullboxwidth for arg 2 of boxfigure to get box of size \textwidth


% \boxalg{pos}{text}:  An algorithm with a box around it
%
% title  the name of the algorithm
% text   the contents of the figure, including picture/caption/label/etc
%
\newcommand{\boxalg}[2]{
            \setlength{\boxwidth}{\fullboxwidth}
            \addtolength{\boxwidth}{.1in}
            \vspace{2ex}
            \framebox[\boxwidth][#1]{
                    \vspace{1ex}
                    \begin{minipage}{\fullinboxwidth}
                        #2
                    \end{minipage}
                    \vspace{1ex}
            }
            \vspace{2ex}
}






% see above
\newlength{\fullboxwidth}
\setlength{\fullboxwidth}{\textwidth}
\addtolength{\fullboxwidth}{-0.5in}

\newlength{\fullinboxwidth}
\setlength{\fullinboxwidth}{\fullboxwidth}
\addtolength{\fullinboxwidth}{-0.5in}





\def\bigO{\mathcal{O}}
\def\xstar{x_{*}}
\def\Rn{\real^{n}}
\newcommand {\real} {\mathbb{R}}
\newcommand {\grad}{\nabla}
\newcommand {\hess}{\nabla^2}

\definecolor{navy}{RGB}{0,0,128}
\definecolor{forestgreen}{RGB}{34,139,34}
\definecolor{mylightsteelblue}{RGB}{176,196,222}
\definecolor{mylightcyan}{RGB}{180,255,255}
\definecolor{mypaleturquoise}{RGB}{175,238,238}
\definecolor{mylightgoldenrod}{RGB}{250,250,210}
\definecolor{mylightyellow}{RGB}{255,255,224}
\definecolor{mylightsalmon}{RGB}{255,160,122}

\setbeamercolor{blacklightsteelblue}{fg=black,bg=mylightsteelblue}
\setbeamercolor{blackpaleturquoise}{fg=black,bg=mypaleturquoise}
\setbeamercolor{blacklightcyan}{fg=black,bg=mylightcyan}
\setbeamercolor{blacklightgoldenrod}{fg=black,bg=mylightgoldenrod}
\setbeamercolor{blacklightyellow}{fg=black,bg=mylightyellow}
\setbeamercolor{blacklightsalmon}{fg=black,bg=mylightsalmon}
\setbeamercolor{blackcyan}{fg=black,bg=cyan}


\begin{document}

%%%%%%%%%%%%%%%%%%%%%%%%%%%%%%%%%%%%%%%%%%%%%
\frame{

\titlepage

} % end frame

\section{Introduction}
%%%%%%%%%%%%%%%%%%%%%%%%%%%%%%%%%%%%%%%%%%%%%
\frame{\frametitle{Short History of Monte Carlo}

\begin{itemize}
\item Original Metropolis algorithm dates back to 1953 paper (J. Chem Phys) for integrals in $\real^{2N}$, where $N$ denoted the number of particles in a problem of interest
\item Hastings proposed a solution to the "curse of dimensionality" in 1970 paper\cite{Hastings} .
\item Gelfand and Smith \cite{GeSm90} (1990) brought more attention to the subject
\item Advent of more computational power, especially parallel processing makes MC methods more popular
\item Combination of computing power and algorithms drives new interest in Bayesian statistics
\end{itemize}
} % end frame

%%%%%%%%%%%%%%%%%%%%%%%%%%%%%%%%%%%%%%%%%%%%%
\frame{\frametitle{Motivating applications}

Example 1: Generate samples from a given probability distribution function (pdf), $f(x)$
$$
X_1, X_2, \ldots, X_t 
$$
Example 2: Computation of certain integrals:
$$
I = \int \! f(x) \, \mathrm{d}x
$$ 
Example 3: Given $f: \Rn \rightarrow \real$
$$
\mbox{find} \ \xstar \in \Rn  \ \mbox{such that} \ f(\xstar) \leq f(x), \ \forall x \in \Rn
$$
} % end frame

%%%%%%%%%%%%%%%%%%%%%%%%%%%%%%%%%%%%%%%%%%%%%
\frame{\frametitle{Some important assumptions \& facts}

\begin{itemize}
\item We will usually assume that we can compute the pdf $f(x) $ only up to a multiplicative constant
\item Don't always know the normalizing constant 
\item Sampling from a given distribution can be quite difficult, especially in the high-dimensional case
\item We should pick more samples from regions where $f(x)$ is "big", but how do we know that without knowing where $f$ is "big"
\end{itemize}

} % end frame

%%%%%%%%%%%%%%%%%%%%%%%%%%%%%%%%%%%%%%%%%%%%%
\frame{\frametitle{Acceptance/Rejection Method}
\begin{algorithm}[H]
\caption{Acceptance-Rejection (AR)}
Given  $f(x) \leq M g(x)$ \\ 
%\begin{algorithmic}[1]
\Repeat{(done)}{
Generate  $X \sim g \mbox{ and } u \sim U(0,1)$ \\
\If {$u \leq f(X) / Mg(X)$}{$Y = X$}\
}
%\end{algorithmic}

\label{alg:AR}
\end{algorithm}

\begin{block}{Remark}
Algorithm (AR) produces a variable Y distributed according to $f(x)$.
\end{block}
} % end frame

%%%%%%%%%%%%%%%%%%%%%%%%%%%%%%%%%%%%%%%%%%%%%
\frame{\frametitle{Integration Example}

Suppose we want to compute the following integral (expectation):
$$
E[f(X)] = \frac{\int \! f(x) \pi(x) \, \mathrm{d}x}{\int \! \pi(x) \, \mathrm{d}x},
$$  and the distribution of $X$ is in $k$-dimensional Euclidian space.
\vspace{.25in}

First we compute samples $\left ( X_t, t = 1, \ldots , N \right )$ with the given distribution $f(x)$ and then approximate
$$
E[f(X)] = \frac{1}{n} \sum_{t=1}^{n} \! f(X_t) .
$$ 

\begin{block}{Remark}
Accuracy will depend on distribution, number of samples we generate, and dimension of the problem. The last is known as the "curse of dimensionality"
\end{block}
} % end frame

 
\section{Metropolis-Hastings}

%%%%%%%%%%%%%%%%%%%%%%%%%%%%%%%%%%%%%%%%%%%%%
\frame{\frametitle{Metropolis-Hastings (1970)}
Let 
$$
\alpha(X,Y) = \min \left ( 1, \frac{\pi(Y) q(X|Y)}{\pi(X) q(Y|X)} \right ) .
$$
\begin{algorithm}[H]
\caption{Metropolis-Hastings (MH)}
Initialize $X_0 $; set $ t =0 $ \\
%\begin{algorithmic}[1]
\While{(not done)}{
Generate  $Y \sim q( \cdot | X_t) ;  u \sim U(0,1)$ \\
\eIf {$u \leq \alpha (X_t, Y)$}{$X_{t+1} = Y$}
{$X_{t+1} = X_t$}
Increment $t$
}
%\end{algorithmic}
\label{alg:MH}
\end{algorithm}

\begin{block}{Terminology}
$q(\cdot|X) $ is called the proposal distribution. 
\end{block}

} % end frame



%%%%%%%%%%%%%%%%%%%%%%%%%%%%%%%%%%%%%%%%%%%%%
\frame{\frametitle{Ising Model Part 1}

Suppose we have a $2D$ array of spins $\sigma_i \in [-1,1]$ and an energy function defined for a configuration of a system
$$
E(\sigma) = - \sum_{ij} {\cal J}_{ij} \sigma_i \sigma_j - B \sum_k \sigma_k
$$
and we want to take the mean of some function $f(\sigma)$:
$$
{\cal F} = \frac{1}{Z(T)} \sum_{\sigma} f(\sigma) \exp( -E(\sigma) / \kappa T),
$$ 
where $Z(T) = \sum_{\sigma}  \exp( -E(\sigma) / \kappa T)$ is the {\it partition function}

} % end frame

%%%%%%%%%%%%%%%%%%%%%%%%%%%%%%%%%%%%%%%%%%%%%
\frame{\frametitle{Ising Model Part 2}

Ideally would like to select configurations from the distribution
$$
g( \sigma) = \frac{\exp( -E(\sigma) / \kappa T)}{Z(T)},
$$ 
and we could approximate the mean by:
$$
F = \frac{\sum_{k} f (\sigma_k)}{M}
$$
\begin{block}{Remark}
Original Metropolis algorithm observed that if only one spin is changed, sampling $g$ is easy. Showed that if a move lowers the energy accept it. If it raises the energy accept it with probability $p$.  Use AR algorithm to choose site for move.
\end{block}
} % end frame

%%%%%%%%%%%%%%%%%%%%%%%%%%%%%%%%%%%%%%%%%%%%%
\frame{\frametitle{Simulated Annealing Method}
\begin{algorithm}[H]
\caption{Simulated Annealing (AR)}

%\begin{algorithmic}[1]
Initialize $X_0 $; set $ t =0 $ \\
\While{(not converged)}{
Generate new state $Y$ and $u \sim U(0,1)$ \\
\eIf {$E(Y) < E(X_t)$}{
$X_{t+1} = Y$
}{
\If {$ u <  e^{- \left [ (E(Y) - E(X_n)) / T \right ]}$}{$X_{t+1} = Y$}
}
}
%\end{algorithmic}

\label{alg:SA}
\end{algorithm}

\begin{block}{Remark}
Simulated Annealing always takes a point where function is lower.  But it will also accept a higher function value with a certain probability depending on the "temperature" $T$.
\end{block}
}
\section{Markov Chain}
%%%%%%%%%%%%%%%%%%%%%%%%%%%%%%%%%%%%%%%%%%%%%
\frame{\frametitle{Basic Markov Chain Facts}

\begin{itemize}
\item A Markov Chain is a sequence of random variables
\item Constructed from a transition kernel K, which is a conditional probability density s.t.
$ X_{n+1} \sim K(X_n, X_{n+1})$
\item Has stability property in that a {\it stationary probability distribution} exists by construction, i.e.  if $ X_n \sim \pi $ then $X_{n+1} \sim \pi $
\item Homogeneous is when transition probability is independent of state -- hence if the initial distribution or initial state is known, then the construction of MC is entirely dependent on transition
\end{itemize}

} % end frame

%%%%%%%%%%%%%%%%%%%%%%%%%%%%%%%%%%%%%%%%%%%%%
\frame{\frametitle{Basic Idea for Markov Chain Monte Carlo}

\begin{itemize}
\item Use properties of Markov Chains to construct a transition kernel that has a stationary probability distribution that matches the one we want
\item Assume stationarity, irreducibility, and aperiodic, we can prove Ergodic Theorem.
\item Needs to also satisfy reversibility condition $ \pi_i p(i \rightarrow j) = \pi_j p(j \rightarrow i) $
\end{itemize}
} %


%%%%%%%%%%%%%%%%%%%%%%%%%%%%%%%%%%%%%%%%%%%%%
\frame{\frametitle{Practicalities}

\begin{itemize}
\item Need to have a "burn-in" period
\item How long do you run the Markov Chain
\item Can use multiple chains to accelerate convergence
\end{itemize}

} % end frame


\section{Summary}
%%%%%%%%%%%%%%%%%%%%%%%%%%%%%%%%%%%%%%%%%%%%%
\frame{\frametitle{Summary}
\begin{itemize}
\item Metropolis algorithm defines a Markov Chain whose limit is the desired probability distribution
\item Many variations on the proposal distribution
\item Growing number of applications
\end{itemize}
}% end frame


%%%%%%%%%%%%%%%%%%%%%%%%%%%%%%%%%%%%%%%%%%%%%
\begin{frame}[allowframebreaks]
  \frametitle<presentation>{References}    
  \begin{thebibliography}{10}    

 \beamertemplatearticlebibitems
 \bibitem{Hastings} W.K. Hastings, \newblock {\em Monte Carlo Sampling Methods Using Markov Chains and Their Applications, }Biometrika, Vol. 57, No. 1., pp. 97-109, 1970

 \beamertemplatearticlebibitems
 \bibitem{GeSm90} A.E. Gelfand and A.F.M. Smith, \newblock {\em Sampling-Based Approaches to Calculating Marginal Densities,} Journal of the American Statistical Association, Vol. 85, No. 410. (Jun., 1990), pp. 398-409

 \beamertemplatearticlebibitems
 \bibitem{BeSu00}
I. Beichi, F. Sullivan
  \newblock The Metropolis Algorithm, Computing in Science and Engineering, January/February 2000.

 \end{thebibliography}
 \end{frame}

\end{document}
%%%%%%%%%%%%%%%%%%%%%%%%%%%%%%%%%%%%%%%%%%%%%
\frame{\frametitle{Metropolis Example}

Suppose we want to compute the following integral:
$$
I = \int \! F (\theta ) \exp \left ( -E(\theta) /kT \right ) \, \mathrm{d}\theta
$$ 
on $\real^{2N}, \theta$ are the positions of $N$ particles on $\real^2$. \\

First we compute samples with the given distribution $f(x)$ and then approximate
$$
I = \sum_{i=1}^{n} \! f(x_i) 
$$ 

\begin{block}{Remark}
Accuracy will depend on distribution, number of samples we generate, and dimension of the problem. The last is known as the "curse of dimensionality"
\end{block}
} % end frame

