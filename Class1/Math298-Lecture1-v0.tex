\documentclass[mathserif,notes]{beamer}
\usepackage{mybeamertheme}
%\usepackage{pdfsync} % works only on TeXShop, comment otherwise
\usepackage{pgf}
\usepackage{multimedia} % to embed movies in the PDF file
\usepackage{xcolor}
\usepackage{graphicx}
\usepackage{comment}
\usepackage{pgfpages}

\mode<presentation>
{
  \usetheme{Madrid}   %  \usetheme{Warsaw}
  % Note: this next command makes nice slides, but Adobe doesn't like it (you get blank pages)
  %\setbeamertemplate{background canvas}[vertical shading][bottom=white,top=structure.fg!15]
  \setbeamertemplate{footline}{\pgfuseimage{UCM-logo}}

  %\usecolortheme{seahorse}
  \usecolortheme{beaver}
 % \setbeamercovered{transparent}
}
%\usepackage[T1]{fontenc}
% For printing purposes only
\pgfpagesuselayout{4 on 1}[letterpaper,border shrink=5mm,landscape]

\usepackage{amssymb,amsmath}
\hypersetup{%
 pdftitle={Math298 Lecture1},
 pdfauthor={Juan Meza(UC Merced)},
 pdfsubject={},
 pdfkeywords={}
 }
 
\title[Math 298]{Math 298 \\
Fundamental Concepts in Computational and Applied Mathematics}  \subtitle{}
\author[Juan Meza]{Juan Meza \\ University of California, Merced}
\date[September 9, 2013]{}
\institute[UC Merced]

\pgfdeclareimage[height=0.35cm]{UCM-logo}{UCM-logo}
% \logo{\pgfuseimage{UCM-logo}}

%% figures






% Common abbreviations
% (Remember to put '\ ' after if an interword space is
%  desired rather than end-of-sentence space. Same for '.etc)' ).
\newcommand{\eg}{{\em e.g.}}		% e.g.
\newcommand{\ie}{{\em i.e.}}		% i.e.
\newcommand{\etc}{{\em etc.}}		% etc.
\newcommand{\vs}{{\em vs.}}		% vs.



\newcommand {\DS} {\displaystyle}
\newcommand{\eref}[1]{{\rm{(\ref{#1})}}}
\newcommand{\tref}[1]{{\rm{\ref{#1}}}}


\def\frechet{Fr\'echet\ }
\def\more{Mor\'e\ }
\def\fl{{\sf fl}}
\def\flop{{\sf flop} }
\def\flops{{\sf flops} }
\def\matlab{{\sc Matlab} }
\def\Matlab{\matlab}



% Mathematical Symbols

\newcommand{\deq}{\raisebox{0pt}[1ex][0pt]{$\stackrel{\scriptscriptstyle{\rm def}}{{}={}}$}}


\newcommand {\half} {\mbox{$\frac{1}{2}$}}  %machine epsilon
\newcommand{\set}[2]{\left\{ #1 \;:\; #2 \right\}}


\newcommand {\macheps} {\mathbf{\epsilon}}  %machine epsilon
\newcommand {\real} {\mathbb{R}}
\newcommand {\nat} {\mathbb{N}}
\newcommand {\compl} {\mathbb{C}}
\newcommand {\diag} {{\mbox{diag}}}
\newcommand {\meas} {{\mbox{meas}}}
\newcommand {\mspan} {{\mbox{span}}}

\newcommand {\pdx} {\frac{\partial}{\partial x}}
\newcommand {\pdt} {\frac{\partial}{\partial t}}
\newcommand {\pdxx} {\frac{\partial^2}{\partial x^2}}
\newcommand {\dx} {\frac{d}{d x}}
\newcommand {\dt} {\frac{d}{d t}}
\newcommand {\dxx} {\frac{d^2}{d x^2}}

\newcommand {\bA} {\mbox{\boldmath $A$}}
\newcommand {\bB} {\mbox{\boldmath $B$}}
\newcommand {\bF} {\mbox{\boldmath $F$}}
\newcommand {\bI} {\mbox{\boldmath $I$}}
\newcommand {\bP} {\mbox{\boldmath $P$}}
\newcommand {\bY} {\mbox{\boldmath $Y$}}
\newcommand {\bZ} {\mbox{\boldmath $Z$}}
\newcommand {\ba} {\mbox{\boldmath $a$}}
\newcommand {\bb} {\mbox{\boldmath $b$}}
\newcommand {\bd} {\mbox{\boldmath $d$}}
\newcommand {\bff}{\mbox{\boldmath $f$}}
\newcommand {\bk} {\mbox{\boldmath $k$}}
\newcommand {\bp} {\mbox{\boldmath $p$}}
\newcommand {\bs} {\mbox{\boldmath $s$}}
\newcommand {\bu} {\mbox{\boldmath $u$}}
\newcommand {\bv} {\mbox{\boldmath $v$}}
\newcommand {\bw} {\mbox{\boldmath $w$}}
\newcommand {\bx} {\mbox{\boldmath $x$}}
\newcommand {\by} {\mbox{\boldmath $y$}}
\newcommand {\bz} {\mbox{\boldmath $z$}}
\newcommand {\blambda} {\mbox{\boldmath $\lambda$}}
\newcommand {\btau} {\mbox{\boldmath $\tau$}}

%
% JCM Quantum Chemistry commands
%
\newcommand{\bra}[1]{\langle #1|}
\newcommand{\ket}[1]{|#1\rangle}
\newcommand{\braket}[2]{\langle #1|#2\rangle}
% Example of usage
%\[
%\ket{\Psi}=\sum_{i}\ket{\phi_i}\braket{\phi_i}{\Psi}
%\]

%
% Other symbols
%
\newcommand {\grad}{\nabla}
\newcommand {\hess}{\nabla^2}
\newcommand {\condA}{\kappa(A)}


\newtheorem{algorithm}{Algorithm}

%

% \boxfig{pos}{wid}{text}:  A figure with a box around it
%
% pos	the usual figure placement arg: eg. htbp
% wid	the width of the figure, in some units: eg. 5in
% text	the contents of the figure, including picture/caption/label/etc
%
\newlength{\boxwidth}
\newcommand{\boxfigure}[3]{
	\begin{figure}[#1]
		\setlength{\boxwidth}{#2}
		\addtolength{\boxwidth}{.1in}

		\centering
		\framebox[\boxwidth]{
			\begin{minipage}{#2}
			#3
			\end{minipage}
		}
	\end{figure}  
}

% use \fullboxwidth for arg 2 of boxfigure to get box of size \textwidth


% \boxalg{pos}{text}:  An algorithm with a box around it
%
% title  the name of the algorithm
% text   the contents of the figure, including picture/caption/label/etc
%
\newcommand{\boxalg}[2]{
            \setlength{\boxwidth}{\fullboxwidth}
            \addtolength{\boxwidth}{.1in}
            \vspace{2ex}
            \framebox[\boxwidth][#1]{
                    \vspace{1ex}
                    \begin{minipage}{\fullinboxwidth}
                        #2
                    \end{minipage}
                    \vspace{1ex}
            }
            \vspace{2ex}
}






% see above
\newlength{\fullboxwidth}
\setlength{\fullboxwidth}{\textwidth}
\addtolength{\fullboxwidth}{-0.5in}

\newlength{\fullinboxwidth}
\setlength{\fullinboxwidth}{\fullboxwidth}
\addtolength{\fullinboxwidth}{-0.5in}






\definecolor{navy}{RGB}{0,0,128}
\definecolor{forestgreen}{RGB}{34,139,34}
\definecolor{mylightsteelblue}{RGB}{176,196,222}
\definecolor{mylightcyan}{RGB}{180,255,255}
\definecolor{mypaleturquoise}{RGB}{175,238,238}
\definecolor{mylightgoldenrod}{RGB}{250,250,210}
\definecolor{mylightyellow}{RGB}{255,255,224}
\definecolor{mylightsalmon}{RGB}{255,160,122}

\setbeamercolor{blacklightsteelblue}{fg=black,bg=mylightsteelblue}
\setbeamercolor{blackpaleturquoise}{fg=black,bg=mypaleturquoise}
\setbeamercolor{blacklightcyan}{fg=black,bg=mylightcyan}
\setbeamercolor{blacklightgoldenrod}{fg=black,bg=mylightgoldenrod}
\setbeamercolor{blacklightyellow}{fg=black,bg=mylightyellow}
\setbeamercolor{blacklightsalmon}{fg=black,bg=mylightsalmon}
\setbeamercolor{blackcyan}{fg=black,bg=cyan}




\begin{document}

%%%%%%%%%%%%%%%%%%%%%%%%%%%%%%%%%%%%%%%%%%%%%
\frame{

\titlepage

} % end frame

\section{Introduction}
%%%%%%%%%%%%%%%%%%%%%%%%%%%%%%%%%%%%%%%%%%%%%
\frame{\frametitle{Course Goals}

\begin{itemize}
\item Introduce fundamental concepts in computational and applied mathematics.
\item Highlight some of the classic NA papers and algorithms.
\item Demonstrate use of these concepts in real world applications.
\item We will not go into any one area in depth.
\end{itemize}
} % end frame


%%%%%%%%%%%%%%%%%%%%%%%%%%%%%%%%%%%%%%%%%%%%%
\frame{\frametitle{Learning Outcomes}

\begin{itemize}

\item Be familiar with key mathematical concepts used in developing numerical algorithms.
\item Understand some of the basic skills and resources necessary to start research in computational and applied mathematics.
\item Be aware of basic communications skills needed to present mathematics clearly to a
  broad audience in writing and in speaking.
\end{itemize}

} % end frame

%%%%%%%%%%%%%%%%%%%%%%%%%%%%%%%%%%%%%%%%%%%%%
\frame{\frametitle{Game Rules}

\begin{itemize}
\item Class participation is critical to getting the most out of this course
\item Some assigned readings
\item Presentations at the end of the semester
\end{itemize}

} % end frame

\section{Preliminaries}
%%%%%%%%%%%%%%%%%%%%%%%%%%%%%%%%%%%%%%%%%%%%%
\frame{\frametitle{Theory vs. Practice}

\begin{itemize}
\item In theory there is no difference between theory and practice. \pause
\item In practice there is.

\end{itemize} 
} % end frame

%%%%%%%%%%%%%%%%%%%%%%%%%%%%%%%%%%%%%%%%%%%%%
\frame{\frametitle{Finite Precision}

\begin{itemize}
\item Difference between infinite precision and computer arithmetic
\item You will always make an error when approximating a mathematical object on a computer
\item One of the major types of errors is discretization(truncation), which is the difference between the solution of the discrete problem and the exact solution of the mathematical object
\item Example: Suppose you have a function defined by an infinite series and you ``truncate" it at some point, e.g.

\begin{eqnarray*}
\exp(x) &=& 1 + x + \frac{x^2}{2!} +  \frac{x^3}{3!} + \ldots \\
&\approx& 1 + x
\end{eqnarray*}

\end{itemize}

} % end frame

%%%%%%%%%%%%%%%%%%%%%%%%%%%%%%%%%%%%%%%%%%%%%
\frame{\frametitle{Finite Precision}

\begin{itemize}
\item Another example

\begin{equation*}
f^{\prime} (x) = \lim_{h \to 0} \frac{f(x+h) - f(x)}{h} 
\end{equation*}
\end{itemize}

\begin{itemize}
\item How do we pick $h$ to get the best approximation? \pause
\item Discussion \pause
\item Lesson: Don't solve an approximate problem too exactly
\end{itemize}

} % end frame

%%%%%%%%%%%%%%%%%%%%%%%%%%%%%%%%%%%%%%%%%%%%%
\frame{\frametitle{Finite Precision Example}

\pgfputat{\pgfxy(0.0,-4.6)}{\pgfbox[left,base]{\pgfimage[width=0.85\textwidth]{Rplot01.pdf}}}

} % end frame

%%%%%%%%%%%%%%%%%%%%%%%%%%%%%%%%%%%%%%%%%%%%%
\frame{\frametitle{Finite Precision}

\begin{equation*}
f^{\prime} (x) = \lim_{h \to 0} \frac{f(x+h) - f(x)}{h} 
\end{equation*}

\begin{itemize}
\item Lesson: Don't solve an approximate problem too exactly
\item Rule of Thumb:  Choose $h$ to perturb about half of the digits
\end{itemize}

} % end frame

%%%%%%%%%%%%%%%%%%%%%%%%%%%%%%%%%%%%%%%%%%%%%
\frame{\frametitle{Finite Precision Real World Examples}

Applications requiring extra precision
\begin{itemize}
\item A
\item B
\item C
\item D
\end{itemize}

} % end frame

%%%%%%%%%%%%%%%%%%%%%%%%%%%%%%%%%%%%%%%%%%%%%
\frame{\frametitle{Conditioning}

\begin{itemize}
\item Definition of conditioning
\item Some examples
\item Rule of thumb
\end{itemize}
} % end frame

%%%%%%%%%%%%%%%%%%%%%%%%%%%%%%%%%%%%%%%%%%%%%
\frame{\frametitle{Stability}
\begin{itemize}
\item Definition of stability
\end{itemize}

} % end frame


\section{Summary}
%%%%%%%%%%%%%%%%%%%%%%%%%%%%%%%%%%%%%%%%%%%%%
\frame{\frametitle{Summary}

\begin{itemize}
\item Beware catastrophic cancellation - know limits of precision
\item Ask about conditioning of a problem
\item Choose stable algorithms so as to not introduce any more loss of precision
\end{itemize}

}% end frame

\end{document}
